\chapter*{Abstract}
\markboth{Abstract}{Abstract}
{\centering
Seeing and Hearing Fluid Subspaces \par
\bigskip\bigskip
by \par
\bigskip\bigskip
Aaron Demby Jones \par
\bigskip\bigskip
}
Fluids have inspired generations of artists and scientists throughout history. Aesthetically, the wide variety of abstract shapes they form is both surprising and pleasing. Besides visual art, which until the digital age mostly captured frozen moments in time, late 19th-century composers such as Debussy and Ravel wrote works of music inspired by the movement of fluids over time. With the framework of several basic conservation laws of physics, earlier 19th-century scientific work discovered a set of differential equations called the {\em Navier-Stokes equations} that described the time evolution of fluid velocity fields. 

In recent years, the advent of higher computing power and the birth of computer graphics as a discipline has given rise to computational methods for approximating and visualizing solutions to the Navier-Stokes equations, which had previously remained intractably complex. Many artists and musicians have also embraced digital technologies, allowing for the development of algorithmically generated music as well as multimodal representations of large, complex data sets. 

With this new technology, it is natural to consider the following question: is it possible to {\em systematically} generate sounds from fluid dynamics while retaining an underlying musicality? In this dissertation, we present a framework for generating correlated correlated fluid motions and musical sounds using the empirical eigenvalues of a subspace fluid simulation. Our method is multimodal in nature, allowing for the generation of musical sound as well as novel visual forms. The specific mapping from fluid velocity to sound chosen allows for control and modulation of both the visuals and the audio in an integrated, unifying fashion.

The method of subspace simulation, which our mapping framework relies on, has a known drawback of high memory consumption. As a means of overcoming this technical obstacle, we also present a data compression framework for fluid subspaces. Our proposed algorithm can achieve an order of magnitude data compression without any noticeable visual artifacts. Using this compression algorithm allows the potential for simulating  greater variety of complex scenes on powerful computers as well as the ability to run previously too-complex scenes on a laptop.