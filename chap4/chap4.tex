\chapter[Transform-based compression of fluid subspaces]{Transform-based compression of fluid subspaces}

In the previous chapter, we discussed the potential for subspace methods to accelerate the computational cost of physics-based simulations. However, a significant drawback of the subspace approach is the time/memory tradeoff: the speed increase comes at a cost of much larger memory requirements. Specifically, subspace simulations can easily consume dozens of gigabytes of memory when dealing with high-resolution scenes. In this chapter, we discuss a compression method to reduce the memory footprint of subspace methods by an order of magnitude. 

\section{Previous Work}
Since memory consumption is a known challenge with subspace techniques, other research has focused on reducing the memory footprint of these simulations. In the applications of sound \cite{Langlois:2014:ECM} and blendshape matrices \cite{Seo:2011:CDM}, compression techniques have been developed; however, we are unaware of analogous research in subspace fluid simulation. In the work of Wicke et al.~\cite{Wicke:2009}, a modular fluid basis is used that can be tiled throughout the domain. However, our approach is complementary, as the modular tiles themselves could be further compressed by applying our algorithm.

\section{Background}
\subsection{Data Compression Preliminaries}
The general problem of data compression at first blush seems daunting: how do we reduce the memory footprint of a set of data without losing information? Fortunately, the sets of data that we typically work with are not random, but rather contain many patterns and redundancies. Exploiting these redundancies is the key to any successful data compression algorithm.

Data compression algorithms can be divided into two distinct families: lossless and lossy. A lossless compression algorithm is able to take an input signal, compress it, and reconstruct the exact same input signal. A simple example is run-length coding. For instance, to represent 100 white pixels, followed by 200 black pixels, followed by an another 50 white pixels, we can code the sequence (`w', 100), (`b', 200), (`w', 50), rather than coding the 350 pixels individually. Decoding is straightforward and retains all information. The ZIP file format is a familiar example of widely-used lossless data compression.

Lossless compression can be very powerful when applied to original data with many redundancies; however, it has strict limitations based on the mathematics of information theory. Every file, no matter how redundant, contains a certain amount of information. \todo{Add details about information theory/entropy.}

Lossy compression, by contrast, may not be able to reconstruct the exact same input signal. Its applications tend to focus on perceptual data, such as images, video, and sound. Although its reconstructions are not perfectly faithful, if the technique is soundly based on the limits of human perception, the results are virtually indistinguishable. The JPEG file format for images is a widely-used lossy data compression algorithm.

Data streams of images oftentimes contain more information than a human can reasonably resolve. For example, a picture might contain fine details that are indistinguishable from their smoothed counterpoints, or subtle color gradients that are indistinguishable from coarser ones. The key to most lossy data compression algorithms is finding a way of transforming the original information into a domain that captures these characteristics more naturally. These techniques are called {\em transform coding}.

\subsection{Mathematical Preliminaries}
In order to describe transform coding formally, we will require several notions from linear algebra and analysis.
\subsubsection{Vector Spaces}
The basic framework of most of our data manipulation can be expressed through the notion of vector spaces. We are familiar with the physical notion of vectors in two- or three-dimensional space as quantities with both a magnitude and a direction, often depicted graphically with an arrow. \todo{Add figure of a 2D vector.} For convenience, we often represent vectors using coordinates. We can view the coordinate representation as a decomposition into fundamental unit vectors pointing in the $x$ and $y$ direction. These are called the {\em basis} vectors. Any arbitrary vector can be decomposed uniquely into a combination of these basis vectors. 

Vectors can be added or subtracted, producing another vector. They can also be multiplied by a scalar (i.e.,~a number in $\R$ or $\C$), yielding another vector. The formalization of this idea is the mathematical concept of a {\em vector space}, which allows for a more abstract treatment of vectors. For example, an image represented on a computer as a sequence of $N$ pixel values can be regarded as a vector living in the vector space $\R^N$. This abstraction allows us to bring the tools of linear algebra to bear on a variety of seemingly different applications.

\subsubsection{Inner Products}
\label{sec:innerprod}
Recall that an {\em inner product}, or dot product, is a binary operation $\langle \cdot, \cdot \rangle$ between two vectors that yields a scalar as an output. The most familiar example is in $\R^2$, where, given two vectors $\vv = (v_1, v_2)$ and $\ww = (w_1, w_2)$, we have $\langle \vv, \ww \rangle = v_1w_1 + v_2w_2$. Inner products add a notion of geometry to a vector space through angles and length: when an inner product between two vectors is $0$, the vectors are orthogonal; and the inner product of a vector with itself gives the square of its length. For example, the basis vectors $\ee_x = (1, 0)$ and $\ee_y = (0, 1)$ are orthogonal, and the vector $\vv = (3, 4)$ has squared length $\langle \vv, \vv \rangle = 3^2 + 4^2 = 5^2$. Inner products also are closely related to the projection of one vector onto another. For example, if we take an arbitrary vector $\vv = (v_1, v_2)$ and take its inner product with the basis vector $\ee_y = (0, 1)$, the result is the component $v_2$ in the $y$ direction. In general, given a set of unit basis vectors which are mutually orthogonal (known as an {\em orthonormal} basis), we can compute the various components of arbitrary vectors by calculating their inner products with the corresponding basis vector.

Besides the familiar inner product in $\R^2$, more general inner products exist in other vector spaces. The canonical inner product in $\C^n$ is given by $\langle \vv, \ww \rangle = \displaystyle \sum_{k=1}^{n}v_k\overline{w_k}$, where the overbar denotes the complex conjugate. The conjugate may seem strange, but is necessary to preserve the notion that taking an inner product of a vector with itself should give the squared length of the vector. The canonical inner product in the function space $L_2([-\pi, \pi])$, the space of functions on the interval $[-\pi, \pi]$ with finite energy, is given by $\langle f, g \rangle  = \displaystyle \int_{-\pi}^{\pi}f(x)\overline{g(x)}dx$.

\subsubsection{The Discrete Fourier Transform}
The {\em Discrete Fourier Transform}, or DFT, is a transformation that maps one vector in $\C^{N}$ to another vector $\C^{N}$. Given a vector $\xx = \left(x_0, x_1, \dots, x_{N-1}\right)$, the DFT maps $\xx$ to the output $\XX = \left(X_0, X_1, \dots, X_{N-1}\right)$, where

\begin{equation}
\label{eq:dft}
	X_k = \sum_{n=0}^{N-1}x_n e^{-i2\pi n k/N}, \ \ k = 0, 1, \dots, N-1
\end{equation}

This definition, as written, appears somewhat unmotivated, so we give a brief geometrical interpretation of the DFT in terms of inner products and roots of unity. 
To begin, we write $\omega = e^{i2 \pi / N}$ as the first $N$-th root of unity\footnote{That is, $\omega$ satisfies $\omega^N = 1$.}. Then the $N$ powers of $\omega$ comprise the entire set of $N$-th roots of unity: $\omega^0, \omega^1, \omega^2, \dots, \omega^{N-1}$. If we collect these into a single vector $\boldomega$, we have 

\begin{equation}
	\boldomega = \left(1, \omega^1, \omega^2, \dots, \omega^{N-1}\right) \in \C^N. 
\end{equation}

We can also consider the $N$ powers of $\boldomega$:
\begin{equation}
	\boldomega^k = \left(1, \omega^k, \omega^{2k}, \dots, \omega^{(N-1)k}\right), \ \ k = 0, 1, \dots, N-1
\end{equation}
The collection of $N$ vectors $\{\boldomega^0, \boldomega^1, \dots, \boldomega^{N-1}\}$ forms an orthogonal\footnote{Orthogonal, but not orthonormal, since each basis vector has magnitude $\sqrt{N} \neq 1$.} basis for $\C^N$, which we shall call the Fourier basis. As such, given an arbitrary vector $\xx = \left(x_0, x_1, \dots, x_{N-1}\right)$, we can compute its representation in the Fourier basis by projecting $\xx$ against each of the new basis functions using the complex inner product. More concretely, the new representation $\XX = \left(X_0, X_1, \dots, X_{N-1}\right)$ in the Fourier basis is given by 

\begin{equation}
	X_k = \frac{1}{\sqrt{N}} \langle \xx, \boldomega^k \rangle, \ \ k = 0, 1, \dots, N-1
\end{equation}
Recalling the definition of the complex inner product given in~\ref{sec:innerprod}, this can be expand to agree with the original definition~\ref{eq:dft} up to a constant factor:

\begin{align}
	X_k &= \frac{1}{\sqrt{N}}\sum_{n=0}^{N-1}x_n \overline{\omega^{nk}}\\
	        &= \frac{1}{\sqrt{N}}\sum_{n=0}^{N-1}x_n \overline{\left(e^{i 2 \pi / N}\right)^{nk}}\\
	        &= \frac{1}{\sqrt{N}}\sum_{n=0}^{N-1}x_n e^{-i2\pi n k/N}, \ \ k = 0, 1, \dots, N-1
\end{align}

This interpretation of projecting against the Fourier basis will prove more useful than the explicit formula for having an intuition about what the DFT is doing under the hood.
In fact, using this interpretation, we can even write the DFT as the following $N \times N$ change-of-basis matrix $\mathcal{F}$, with each of the $N$ columns given by the corresponding $\boldomega^k$ vector:

\begin{align}
	\mathcal{F} &= \begin{pmatrix}
    		\vertbar & \ \  \vertbar & \ \ \ \ \ \ \ \ \  \vertbar & & \vertbar \\
    		\boldomega^0  & \ \   \boldomega^1  & \ \ \ \ \ \ \ \ \  \boldomega^2 & \ \ \ \ \ \dots & \ \ \ \ \ \ \boldomega^{N-1} \\
 		\vertbar & \ \  \vertbar & \ \ \ \ \ \ \ \ \  \vertbar & & \vertbar 
  	\end{pmatrix} \\	
	&= \begin{pmatrix}
	1 & 1 & 1 & \dots & 1 \\
	1 & \omega^{-1} & \omega^{-2} & \dots & \omega^{-(N-1)} \\
	1 & \omega^{-2} & \omega^{-4} & \dots & \omega^{-2(N-1)} \\
	\vdots & \vdots & \vdots & \ddots & \vdots \\
	1 & \omega^{-(N-1)} & \omega^{-2(N-1)} & \dots & \omega^{-{(N-1)}^2}
	\end{pmatrix}	
\end{align}

Thus, the DFT of a vector $\xx \in \C^N$ can now simply be defined by the matrix-vector multiplication $\mathcal{F} \xx = \XX$.

The multidimensional DFT applies takes multidimensional arrays of inputs rather than vectors. For example, the two-dimensional DFT of the array $\xx \in \C^{M \times N}$ is given by
the array $\XX \in \C^{M \times N}$ whose entries are
\begin{equation}
	\XX_{u,v} = \sum_{m=0}^{M-1} \sum_{n=0}^{N-1} \xx_{m, n} e^{-i 2 \pi \left(um + vn\right)}
\end{equation}

In the most general case, given a $d$-dimensional array $\xx \in C^{N_1 \times N_2 \times \cdots \times N_d}$, its d-dimensional DFT is given by the array $\XX \in \C^{N_1 \times N_2 \times \cdots \times N_d}$ whose entries are
\begin{equation}
	\XX_{u_1, u_2, \dots, u_d} = \sum_{n_1=0}^{N_1-1} \sum_{n_2=0}^{N_2-1} \cdots \sum_{n_d = 0}^{N_d -1} \xx_{n_1, n_2, \dots, n_d} e^{-i 2 \pi \left(u_1 n_1 + u_2 n_2 + \cdots + u_d n_d\right)}
\end{equation}
In practice, however, we will stick to using $d=2$ or $d=3$.
\subsection{The JPEG Coding Algorithm}
The JPEG coding scheme is a form of lossy, transform compression for digital images. Because our subspace fluid compression scheme is based on many of the techniques used in JPEG, we shall explore the implementation of JPEG in some detail to better motivate our own coding algorithm. 

The point of departure in JPEG is a digital image file, which comprises a two-dimensional array of pixel values. (For simplicity, we will assume a grayscale image.) The basic redundancy that JPEG exploits is that for most images, the majority of the `energy' is packed into lower spatial frequencies. In other words, generally speaking, there are not many extremely sharp changes between hues in a typical image. (Notable exceptions include medical imaging, for which JPEG is typically not applied.) 

So, to begin, suppose we have a two-dimensional grayscale image of resolution $800 \times 800$. While over the space of the entire image, there may be sharp changes between hues, if we subdivide the image into smaller regions, within each individual region, there are sharp changes much more rarely. Hence, the first step of JPEG is to subdivide the image into many smaller regions. The typical `sweet spot' chosen is $8 \times 8$ blocks, so in this case, there will be $100 \cdot 100 = 10 000$ such blocks. 

To exploit the redundancy of the image in the frequency domain, we now transform each block according to a two-dimensional discrete cosine transform (2D-DCT). The discrete cosine transform, with some massaging, can actually be regarded as a particular case of the discrete Fourier transform, so conceptually, we can regard it in the same spirit. The multidimensional version is simply a separable product of the 1D transforms, and can be computed by first taking the 1D transform across each column and then across each row (the order is irrelevant).

Next, we {\em dampen} the frequencies by point-wise dividing the resulting set of values by a pre-determined damping array, $\QQ$. The goal of this procedure is to dampen enough of the frequencies to values small enough that, upon integer rounding, they become zero. When we scan through the values and flatten them into a 1D vector, each long string of zeroes can be efficiently compressed by run-length coding.

\begin{equation}
	\QQ = \begin{pmatrix}
	16 & 11 & 10 & 16 & 24 & 40 & 51 & 61\\
	12 & 12 & 14 & 19 & 26 & 58 & 60 & 55\\
	14 & 13 & 16 & 24 & 40 & 57 & 69 & 56\\
	14 & 17 & 22 & 29 & 51 & 87 & 80 & 62\\
	18 & 22 & 37 & 56 & 68 & 109 & 103 & 77\\
	24 & 35 & 55 & 64 & 81 & 104 & 113 & 92\\
	49 & 64 & 78 & 87 & 103 & 121 & 120 & 101\\
	72 & 92 & 95 & 98 & 112 & 100 & 103 & 99	
	\end{pmatrix}
\end{equation}











