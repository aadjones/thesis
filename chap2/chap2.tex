\chapter[Related Work]{Related Work}
\label{chap:chap2}
\section{Sonification and Generative Art}

The question of mapping computational fluid dynamics data into sound belongs 
to the intersection of the domains of {\em sonification} and {\em generative 
art}. Sources vary in agreement on the definition of sonification. According 
to Hermann, sonification is ``the technique of rendering sound in response to 
data and interactions.'' \cite{hermann2011sonification} Kramer et al.~define 
it as ``the use of nonspeech audio to convey information.'' \cite{kramer2010sonification} 
We shall make a distinction between {\em scientific} 
sonification, which aims primarily toward conveying information clearly, and {
\em musical sonification}, which aims primarily toward aesthetically useful 
generation of music. There are many possible strategies for sonification, 
including audification, parameter mapping sonification, and model-based 
sonification \cite{hermann2011sonification}. We shall expand upon the meaning 
of these terms in \S\ref{sec:background}.

Generative art is sometimes thought of in very general terms. For instance, 
Boden specifies eleven different categories of generative art: electronic 
art, computer art, computer-assisted art digital art, generative art, computer-generated art, evolutionary art, robot art, interactive art, computer-interactive art, and virtual reality art \cite{boden2009generative}. In the 
text, however, we prefer a more narrow definition of generative art as art 
which has been created through the design and use of a computational system.

As Curtis Roads observes in \cite{roads2015composing}, one of the main attractions of this compositional style is the possibility of 
novel discoveries that go beyond what the artist may otherwise have been able 
to conceive: ``The computer can allow a composer to write music that goes 
beyond that which she is already capable of.'' 

\subsection{Background}
\label{sec:background}
The use of natural processes as a technique in music composition has been 
well known for centuries, from the Greek's use of harmonic proportions in 
tuning to Mozart's dice music. The twentieth century saw an increased 
interest in formalized mathematical systems as part of musical composition. 
Composers and theorists such as Schillinger, Stockhausen, Xenakis, Cage, 
Lucier, Lewin, and many others explored different systematic ways of 
generating sonic art \cite{schillinger1949schillinger, stockhausen1962concept, xenakis1992formalized, cage2011silence, lucier1998origins, lewin2010generalized}.
 In the twenty-first century, with the explosion of the 
idea of ``big data,'' music and sound has been generated from all sorts of 
data sets such as astronomical measurements, seismographs, web traffic, and 
more. Gradually, an important question emerged: is the composer focused 
primarily with the listener's musical experience, or is it the intent that 
the music inform the listener of features of the underlying data? Such a 
question touches the tip of the iceberg of many challenging philosophical 
questions, such as whether music can even have any external meaning, or the 
distinction between the artist's intention and the listener's response. While 
we shall only explore a few of these ideas in some detail, the interested 
reader can find more thorough discussions in \cite{demers2010listening}.

In the earlier forms of sonification, the data itself tended to be the most 
prominent feature. Indeed, sonification was viewed as an offshoot of data 
visualization. Thus, for example, the technique of audification, in which 
data values are mapped directly to sound pressure levels, by maintaining a 
literal connection between the data and the sound, serves as a direct form of 
sonification. Audification has been used in especially for natural phenomenon 
that generate time series, such as seismology, or stock market prices. 
Parameter mapping, by contrast, is more of a second-order technique in which 
one parameter of the data is mapped to a parameter of sound, such as pitch, 
or volume. For example, a series of data representing temperature values 
might be mapped to pitch in such a way that higher temperature sounds as 
higher pitch, and vice versa. Although the data is not literally being 
interpreted as a waveform, the analogy of human perception still makes the 
connection between the data and sound tangible. Finally, more abstract forms 
of sonification, such as model-based techniques, in which a time-varying 
process is devised that connects the data and sound together, can strain the 
perceptual limits of the link between the data and the sound, although 
formally speaking, it still is present. Kramer proposes a ``semiotic'' 
spectrum from analogic to symbolic sonifications, with audification at the 
extreme analogic end as closest to the data, parameter mapping in the middle, 
and model-based techniques at the symbolic extreme \cite{hermann2011sonification}.

\subsection{Model-Based Sonification}
\label{sec:modelbased}
As our particular research strategy resembles the category of model-based 
sonification (MBS) the most closely, we review this technique in some detail.
According to Hermann, ``Model-Based Sonification is a sonification technique 
that takes a
particular look at how acoustic responses are generated in response to the 
user's actions, and
offers a framework to govern how these insights can be carried over to data 
sonification. As
a result, Model-Based Sonification demands the creation of processes that 
involve the data in
a systematic way, and that are capable of evolving in time to generate an 
acoustic signal.'' \cite{hermann2011sonification} The latter quality of 
evolving in time to generate an acoustic signal interests us the most, as the 
time evolution in such models is often governed systematically according to 
physical principles, making it an ideal candidate for a sonification of a 
physical process such as fluid dynamics. After exploring some fundamental 
ideas in human perception and modes of listening, Hermann concludes that a 
successful sonification should satisfy five properties, paraphrased thusly:
\begin{itemize}
	\item Ubiquity: Each interaction with data should produce a sound.
	\item Invariance of binding mechanism: The laws producing the sound from 
the data should not depend on the data itself, much in the same way that the
	laws of physics do not depend on the specific objects that they govern.
	\item Immediate response: Each interaction with the data should produce an 
immediate response in order to mirror as closely as possible interactions 
with objects in the real world
	\item Sonic variability: Sonifications should vary according to subtle 
changes in state and input.
	\item Information richness: Sonifications should be nontrivial.
\end{itemize}
Essentially, these properties guarantee that a sonification mimics 
interactions in the real world with data and sound. This ties in with a more 
analytical mode of listening---for instance, shaking a wrapped birthday 
present to try to determine its contents from the corresponding sound, rather 
than listening to the sound of the jostling present as a musical event. By 
using the technique of MBS, Hermann concludes that these properties will in 
fact automatically be satisfied. The basic principle, thus, is to design 
something analogous to the laws of physics, but as a sonification system of 
data. The model then becomes the rules by which the data is mapped to the 
sound and unfolds over time, much as the laws of physics determine the rules 
by which objects interact with one another in space and time.

Hermann also identifies six key components that aid in the design of a MBS 
system: setup, model dynamics, excitation, initial state, link-variables, and 
listener characteristics. It is also useful to keep as separate abstract 
concepts the data space in which the data lives, the model space in which 
the sonification ``laws'' live, the sound space in which the actual sound 
itself lives, and the listener space in which the listener's reaction to the 
sound lives. The model setup, thus, takes data from the data space, and maps 
it to the space of a dynamical model with time-varying components that create 
the corresponding sound in the sound space. 

As a general example, suppose that the data can be interpreted as a 
collection of vectors $\{\xx_j\}, j=1,\dots,N$, living in the $d$-dimensional 
vector space $\R^d$. (For instance, more concretely, the data might comprise 
the temperature, barometric pressure, and humidity in Santa Barbara for each 
day in the year 2017, meaning $d=3$ and the collection has $N=365$ elements.) 
Such a re-interpretation of the data abstractly in a mathematical space is 
the primary goal of the model setup. The advantage of such an approach is the 
ubiquity of mathematical tools in linear algebra at our disposal if we view our data through the lens of a vector space.

Given our model setup, we next turn to the model dynamics. Since our goal is 
to drive a sound-based model over time, it is natural to consider a time-varying evolution to our model. In other words, we consider the model not as 
a static configuration but as a dynamical system. Mathematically speaking, if 
we write ${\bolds}(t)$ to denote the state of the model at time $t$, then we 
would like a differential equation to govern the time evolution---e.g.,

\begin{equation}
\frac{d{\bolds}}{dt} = f({\bolds}(t), t)
\end{equation}

The function $f$ here determines the model dynamics. Already the connection 
with physics is tempting. Although the function $f$ in principle is 
arbitrary, the ideas of conservation of energy, gravity and buoyancy, 
friction, dissipation, and other physical principles provide an attractive 
palette from which to borrow. 

The excitation can be thought of as analogous perturbations to the balance of 
forces in a dynamical system. In a general dynamical system, an excitation 
might be obtained through interaction, as a user adds external forces to the 
physics through a keystroke or mouse click. For example, in a fluid 
simulation, a user might add a force to the fluid velocity field by clicking 
and dragging the mouse through a region. Other possible fluid examples 
include altering the fluid's viscosity or introducing a buoyant external 
force. 

The model setup is analogous to the initial and boundary conditions of a 
dynamical system. For example, a typical first-order ordinary differential 
equation $\frac{dx}{dt} = f(x(t), t)$ defines an entire family of possible 
solutions $x(t)$. Only by specifying an initial condition $x(0) = x_0$ do we 
uniquely determine which trajectory to take. More general differential 
equations, in addition to requiring initial conditions, may also require 
boundary conditions to determine edge behaviors. For instance, in a fluid 
simulation, we frequently use the ``no-slip'' boundary condition, meaning 
that at a solid boundary, the fluid has zero velocity relative to the 
boundary \cite{day1990no}. These parameters can greatly influence the resulting simulation, 
but the designer typically through experience and experimentation has 
knowledge of which conditions to use in order to obtain the desired physical 
effect.

Link variables form the bridge between the dynamical process of the model and 
actual sound. In the physical world, these connections often happen 
literally. For instance, if we model the rigid vibrations of a metal plate 
using a spring-mass system, the corresponding air pressure variations it 
induces lead directly to an audible sound signal. Naturally, this isn't so 
much of a sonification as an actual physical calculation of sound, but the 
analogy remains useful for more abstract sonifications. In practice, link 
variables are often more of a second-order connection, much as in parameter-
mapping sonification. For example, we might map the overall energy of a 
system to a musical loudness. A common challenge with MBS rears its head 
here, as depending on the complexity of the model, a real-time sound output 
may not be feasible. In particular, if the link variables are conceived of at 
an audio sample rate, in order to maintain high-quality sound rendering, we 
require at least 44100 samples to be computed per second, which may be 
computationally infeasible. Indeed, the audiovisual system proposed in this 
dissertation cannot be run in real-time using the available computing power 
as of writing in 2017. The tradeoffs of real-time vs. nonreal-time 
composition will be discussed further in Chapter \ref{chap:chap6}.

Listener characteristics aim to understand better the desired mode of 
listening for which the composer aims. For example, shall the listener 
perceive herself to be ``inside'' the system, or an outside observer? Should 
the sound be perceived as a single source, or an entire soundscape? Many of 
these questions have design answers in the form of the spatialization of the 
audio output. For instance, by using careful panning techniques, the composer 
can create many different desired perceptions in the listener.

\subsection{Parameter Mapping Sonification}
\label{sec:PMSon}
Following Grond and Berger in \cite{hermann2011sonification}, Parameter Mapping Sonification (PMSon) is another important technique for 
sonification. It is a step removed from the more literal form of 
audification, in which the raw data is mapped directly into an audio-rate 
signal. Instead, PMSon is characterized by a mapping of abstract information 
to auditory parameters. For instance, we might map the value of a stock, as a 
piece of abstract information, to a frequency of a sound, as an auditory 
parameter. While more indirect than audification, working more by analogy 
than by literal mapping, PMSon can be an especially useful choice for 
multidimensional data, in which different facets of the data can then be 
mapped to different facets of the audio.

Many challenges arise in PMSon. For instance, without the direct 
interpretation offered by audification, there is no one `correct' way to make 
mapping associations. Indeed, in most PMSon systems, a wide range of 
interpretation is possible, even allowing for multiple different 
sonifications of the same underlying data. Another challenge is that of human 
perception: many data sets will generate output signals that exceed the 
limits of human hearing, either through resolution (just-noticeable 
differences) or through bandwidth (absolute thresholds). Thus, any 
practically useful mapping must take these limitations into effect, generally 
by adapting or scaling the data accordingly. The choice of which audio 
features to use remains an open question as well, as there are many from 
which to choose: pitch, timbre, rhythm, texture, envelope, duration, etc. 

\subsubsection{Formalization via Transfer Function}
As a mathematical formalism for speaking in general terms about PMSon, 
consider a $d$-dimensional data set $\{\xx_1, \dots, \xx_N\}, \xx_j \in \R^d$
. Let the sound parameter space have $n$ dimensions, so it is identified with 
$\R^n$. (It needn't be the case that $d = n$ in general, hence the use of 
multiple variables.) Next, define $g \colon \R^d \rightarrow \R^n$ to be the 
parameter mapping function that maps an element in the data space to an 
element in the $m$-dimensional sound parameter space.  Finally, define $f 
\colon \R^{n + 1} \rightarrow \R^q$ to be the signal generation function, 
which takes as an input an $n$-dimensional vector $\aaa$ of sound parameters 
and a time $t$, and outputs a $q$-channel sound signal $s(t) = f(\aaa; t)$. A 
PMSon is then computed by

\begin{equation}
s(t) = \sum_{j=1}^N f(g(\xx_j); t)
\end{equation}

The mapping function $g$ is the heart of the system. There are several 
different types of mappings: one-to-one, one-to-many, and many-to-one. In a 
one-to-one mapping, a single data point is mapped to a single sound point. 
For example, a temperature reading could be mapped to a sound frequency. In a 
many-to-one mapping, a form of dimension reduction occurs in the data. As a 
simplified example, an average of many temperature readings could be mapped 
to a single sound frequency. Forms of data reduction such as principle 
component analysis (PCA) are commonly used, which forms a nice bridge between 
sonification design and the computer graphics subspace approach we use in our 
physics-based simulations. Finally, one-to-many mappings can arise when a 
single data point maps to several different sound parameters. For instance, 
if a data value of temperature is in a certain region, we might map it to 
both sound frequency and sound amplitude. The multitude of possibilities 
again raises many questions and challenges for the appropriate design of 
PMSon systems.

\subsection{Synthesis Methods}
\label{sec:synthesis-metthods}
In designing any sonification system, the craft of audio synthesis itself is 
an important ingredient. Many different techniques exist for generating 
digital sound. We explore just a handful of possibilities in this section, following Cook in \cite{hermann2011sonification}.

The technique of additive synthesis is perhaps the most basic and 
fundamental. The general strategy is attributed to the concepts of harmonic 
analysis. According to the theory of Fourier series \cite{katznelson2004introduction}, we know that any 
periodic signal $f(t)$ with period $L$ can be decomposed into a Fourier 
series via the equation

\begin{equation}
f(t) = \sum_{n \in \mathbb{Z}} c_n e^{i\frac{2\pi}{L}nt}
\end{equation}

If we write $\omega_0$ as the frequency $\frac{2\pi}{L}$, then we see that $f$
 has been decomposed into a superposition of differently-weighted complex 
exponentials whose frequencies are integer multiples of $\omega_0$. We often 
refer to $\omega_0$ as the {\em fundamental frequency}. 

The technique of additive synthesis is thus to take only pure sinusoidal tones
and mix them together, building up from a fundamental frequency. The higher
frequency tones in the signal, called the {\em partials}, may or may not have
an integer ratio to the fundamental, characterizing the perception of the sound
as harmonic or inharmonic. Besides the ratio between the partial and the fundamental,
the time-varying amplitude envelope of each partial also determines an important
synthesis parameter.

Another important technique of synthesis, modal synthesis, is of particular interest. 
Modal synthesis is based on the concept of a simple harmonic oscillator with damping.
The solution to this differential equation is known to be a cosine wave, dampened by
a decaying exponential term. More precisely, the system in question, generated from
a simple application of Netwon's second law $F = ma$, is given by 

\begin{equation}
m\ddot{y} = -c \dot{y} - ky
\end{equation}

or, rearranging,

\begin{equation}
\label{eq:dampedosc}
\ddot{y} + \frac{c}{m} \dot{y} + \frac{k}{m}y = 0
\end{equation}

Here, the unknown $y = y(t)$ represents displacement from the equilibrium, $m$ is the mass, $c$ is the damping term, $k$ is the spring constant,
and an overdot denotes a time derivative. The solution is given by

\begin{equation}
y(t) = y_0 e^{-\frac{c}{2m}t} \cos\left({t \sqrt{\frac{k}{m} - \left(\frac{c}{2m}\right)^2}}\right)
\end{equation}

While these equations are a bit abstract, they apply directly to simple idealized musical instruments. For example,
the vibrational modes of a plucked string can be determined through such an analysis. The boundary conditions
of the string (assuming it is taut and pinned down at both ends) force the string into a particular set
of modal vibration shapes that correspond to natural frequencies of the string. In a slightly
more sophisticated example, the membrane of a drumhead can also be excited into forming a set of
modal vibration shapes that correspond to its natural frequencies. These frequencies again will depend on the
boundary condition (how the drum is pinned down) as well as the geometric shape of the drumhead. 

Cook presents an interesting interpretation of the dampened harmonic oscillator as a second order,
2-pole feedback filter. If we interpret Equation \ref{eq:dampedosc} using finite-difference approximations for the derivatives, we obtain 

\begin{equation}
\frac{y[N] - 2y[N-1] + y[N-2]}{T^2} + \frac{c}{m} \frac{y[N] - y[N-1]}{T} + \frac{k}{m} y[N] = 0
\end{equation}

Here, $y[N - k]$ denotes the value of the sample $k$ samples ago. Gathering together like terms, we obtain

\begin{equation}
y[N] \frac{m + cT + kT^2}{m} - y[N-1] \frac{2m + cT}{m} + y[N-2] = 0
\end{equation}

Solving for $y[N]$ in terms of the previous samples, we get

\begin{equation}
\label{eq:linsys}
y[N] = c_1 y[N-1] + c_2 y[N-2]
\end{equation}

where $c_1 = \frac{2m + cT}{m + cT + kT^2}$ and $c_2 = \frac{m}{m + cT + kT^2}$. In the Digital
Systems Processing (DSP) literature \cite{smith1997scientist}, Equation \ref{eq:linsys} can be implemented using a second order 2-pole feedback filter. 
Thus, we can interpret modal synthesis as a form of subtractive synthesis, in which a spectrally rich input source, such as
noise, or an impulse, excites modal filters into resonance.

\subsection{Aesthetics}
A continual dilemma in generative art is the conflict between the level of 
rigor of the underlying formal system and the human perception of its output. 
Some artists (e.g. Milton Babbitt) take the dogmatic position that the logic 
of the system trumps the general perception of the output \cite{babbitt1958cares}.
However, others such as K{\v{r}}enek have taken a more careful middle 
ground, arguing in \cite{kvrenek1939music} that the existence of an aesthetically coherent system of 
rules is no guarantee that an aesthetically coherent artistic result will 
perceptibly emerge: ``We cannot take the bare logical coherence of a musical `axiomatic' system as the sole criterion of its soundness! \dots The 
outstanding characteristic of music [is] its independence from the linguistic 
limitations of general logic.'' In this dissertation, our attitude aligns more closely to K{\v{r}}enek than to Babbitt. While the rigor of the underlying system
is an important concern, the sonic output still must be considered carefully and adjusted as necessary. To that end, our freedom of choice in the sonification
strategy allows us to make choices that emphasize musical qualities such as texture, timbre, and dynamic control while still adhering to the logic of the system. We view
our system as a high-level tool through which novel sounds and visuals can be generated automatically, evolving together over time. Aesthetically, the visual and sonic
impression on the viewer/listener is an important criterion which cannot be ignored by appealing to the logic of the system. As a basic example of applying that principle in our work, we selected a
 visually interesting training set of fluid motions so that the subspace dynamics retained a variety of pleasing forms and shapes. A more in-depth discussion of our aesthetic principles as applied
 to our audiovisual system follows in Chapter \ref{chap:chap6}.