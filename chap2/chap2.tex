\chapter[Related Work]{Related Work}
\section{Sonification and Generative Art}

The question of mapping computational fluid dynamics data into sound belongs to the intersection of the domains of {\em sonification} and {\em generative art}. Sources vary in agreement on the definition of sonification. According to Hermann, sonification is ``the technique of rendering sound in response to data and interactions.'' \cite{hermann2011sonification} Kramer et al.~define it as ``the use of nonspeech audio to convey information.'' \cite{kramer2010sonification} We shall make a distinction between {\em scientific} sonification, which aims primarily toward conveying information clearly, and {\em musical sonification}, which aims primarily toward aesthetically useful generation of music. There are many possible strategies for sonification, including audification, parameter mapping sonification, and model-based sonification. \cite{hermann2011sonification} 

Generative art is sometimes thought of in very general terms. For instance, Boden specifies eleven different categories of generative art: electronic art, computer art, computer-assisted art digital art, generative art, computer-generated art, evolutionary art, robot art, interactive art, computer-interactive art, and virtual reality art. \cite{boden2009generative}