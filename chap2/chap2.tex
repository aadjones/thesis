\chapter[Related Work]{Related Work}
\section{Sonification and Generative Art}

The question of mapping computational fluid dynamics data into sound belongs to the intersection of the domains of {\em sonification} and {\em generative art}. Sources vary in agreement on the definition of sonification. According to Hermann, sonification is ``the technique of rendering sound in response to data and interactions.'' \cite{hermann2011sonification} Kramer et al.~define it as ``the use of nonspeech audio to convey information.'' \cite{kramer2010sonification} We shall make a distinction between {\em scientific} sonification, which aims primarily toward conveying information clearly, and {\em musical sonification}, which aims primarily toward aesthetically useful generation of music. There are many possible strategies for sonification, including audification, parameter mapping sonification, and model-based sonification. \cite{hermann2011sonification} 

Generative art is sometimes thought of in very general terms. For instance, Boden specifies eleven different categories of generative art: electronic art, computer art, computer-assisted art digital art, generative art, computer-generated art, evolutionary art, robot art, interactive art, computer-interactive art, and virtual reality art. \cite{boden2009generative}. In the text, however, we prefer a more narrow definition of generative art as art which has been created through the design and use of a computational system.

One of the main attractions of this compositional style is the possibility of novel discoveries that go beyond what the artist may otherwise have been able to conceive: ``The computer can allow a composer to write music that goes beyond that which she is already capable of.'' \cite{roads2015composing}

\subsection{Aesthetics}
A continual dilemma in generative art is the conflict between the level of rigor of the underlying formal system and the human perception of its output. Some artists (e.g. Milton Babbitt) take the dogmatic position that the otic of the system trumps the general perception of the output. \cite{babbitt1958cares}  However, others such as K{\v{r}}enek have taken a more careful middle ground, arguing that the existence of an aesthetically coherent system of rules is no guarantee that an aesthetically coherent artistic result will perceptibly emerge: ``We cannot take the bare logical coherence of a musical `axiomatic' system as the sole criterion of its soundness! \dots The outstanding characteristic of music [is] its independence from the linguistic limitations of general logic.'' \cite{kvrenek1939music}