\chapter[Conclusions and future work]{Conclusions and future work}
\label{chap:chap7}

In this dissertation, toward the goal of developing a generative audiovisual compositional language, we have presented a sonification system for computational fluid dynamics
based on the method of subspaces. Additionally, we have devised a data compression algorithm that alleviates the memory costs of subspace algorithms by an order of magnitude. Through
systematic experimentation and exploration, our sonification system was tested and explored, producing a series of audiovisual etudes. The data compression algorithm was rigorously tested 
on a variety of different types of scenes, perturbations, and compression levels, demonstrating its general robustness and effectiveness. Continuing research in the direction of the sonification
system is an important step in the development of new audiovisual grammars in media art, while further work in the area of data compression represents an important step toward
making subspace simulations more computationally practical.

 \section{Summary of Results}
 We have devised a system of sonification of computational fluid dynamics by considering the empirical eigenvectors and eigenvalues of a subspace simulation. Based loosely on the ideas of 
 Chladni plates and the general phenomenon of cymatics, we map the characteristic resonant modal shapes of the fluid velocity field bases to their corresponding audio frequencies,
 producing a sound signal. The choice of a model-based sonification strategy allows us to unfold our visuals and sounds easily over time within the framework of the system. Hence, the natural
 sequence of fluid velocity fields as governed by the subspace Navier-Stokes equations determines a natural time progression of dynamic visual forms and sonic content. 
 
 Several sound synthesis strategies and transformations had to be considered. We carefully mapped the raw singular values into frequencies,
 choosing an intuitive polarization as well as a ratio-preserving transformation and offset to keep the values within the limits of human hearing. Following the spirit of physical modal vibrations, we 
 selected a subtractive synthesis technique, in which a filterbank of resonant filters at the corresponding modal frequencies were excited in various strengths and decay times. Although many possible
 input signals could be used with the filter bank, we chose to use noise, as it was both spectrally rich and matched the aesthetic feel of fluid flow.

Our audiovisual etudes each explored different combinations of modal excitations. By isolating individual modes, superimposing them, and cross-fading between them, we demonstrated
the versatility and compositional usefulness of having spectral control in both the spatial and audio domain. The gestures of crescendo and diminuendo were also performed by controlling the
overall energy of the system, demonstrating the potential for different musical articulations. Finally, the general principle of constructing time-evolving paths ungoverned by physics illustrated
an important compositional possibility of working with fluids and sounds more fancifully, leading to the idea of smoke ``as it might be'' as opposed to smoke ``as it is.'' 

The data compression technique developed in Chapter \ref{chap:chap4} serves to alleviate the memory pressure of subspace simulations. In an analogy to the JPEG compression scheme,
we use a DCT-based transform compression algorithm to represent the subspace basis vectors in the Fourier domain, with the intuition that the energy at the higher frequencies can be 
dampened or discarded without distorting the original data greatly. Our algorithm, while intuitively based on JPEG, made several important alterations that were necessary for fluid data, including
adapting the process to three dimensions, systematically generating damping quality matrices, performing an energy-based per-block quality selection, and devising a new zigzag scan. In addition, 
since na\"ive reconstruction during the decompression stage required heavy computational costs, largely negating the advantage of the subspace approach to begin with, we discovered a novel 
fast sparse frequency-domain reconstruction, enabling the algorithm to reduce memory costs without severely increasing time costs.

\section{Limitations}
There are several drawbacks to our current sonification and audiovisual system. The most challenging is that of computational speed. While such a system would ideally run in real time,
allowing not only for simple compositional feedback and iteration but also for user interaction, both the simulation and rendering times as of 2017 are extremely far away from these speeds.
This means that even short compositions can take hours or days to generate at high visual quality. However, the restriction of working in non-real time does force the composer to make careful
compositional choices ahead of time, producing aesthetically very different works from real-time, experimental, interactive systems. 

Another basic drawback of the sonification system is inherent in the subspace method itself. Once a training set is chosen and the subspace basis is computed, simulations are locked into 
reproducing only dynamics that are reasonably similar to the original training set. Hence, if the user desires any significantly different set of motions or dynamics, a fresh simulation must be precomputed.
While in principle this can be done repeatedly, the time costs are quite prohibitive. A more analytical approach such as using a particular mathematical basis such as Laplacian eigenfunctions is possible,
but the fluid modes would then be more regular, undermining the visual variety of the training-based approach.

The data compression scheme, while obtaining an order of magnitude compression, still leads to a slight increase in time costs, which may be unacceptable to a user who already
has sufficient memory to compute the uncompressed subspaces. Furthermore, the system assumes a fluid simulation technique using a regular grid, making it unusable for tetrahedral meshes or other
more general fluid simulation methods.

\section{Future Work}
We have explored several basic parameter modulations in our sonification system, producing a series of audiovisual etudes. However, further work in this direction could be considered. Other
parameters such as fluid viscosity or buoyant forces could be modulated, leading to new insights into the interplay between form and sound. The sonifications presented
in this dissertation all derived from one particular subspace training; however, other data sets are possible to generate, leading to different audiovisual links. The introduction
of obstacles would also perturb the basis functions, leading to a novel spectrum. The present rendering system remains static, presenting the smoke in each simulation
with the same lighting and coloring; however, a mapping between the rendering system and sound could also be explored, yielding more variety in the visuals. Finally, more complete audiovisual works,
deeper in macrostructure and musical form, remain to be seen. 

In the data compression scheme, no matter what the simulation, the approach relies on using the discrete cosine basis, not leveraging the potential for more sparse representations depending
on the simulation at hand. More general techniques, such as dictionary-based methods using orthogonal matching pursuit, could lead to sparser representations, and therefore both better memory compression
as well as time reduction, due to the reliance of the algorithm on the sparse frequency-domain reconstruction during the decompression phase. Additionally, besides the run-length encoding step, no further
lossless compression is applied to the data stream during the compressor. However, preliminary tests show that the data stream typically contains informational redundancies, implying that an additional step through
entropy encoding such as Huffman or arithmetic coding could increase the compression efficiency.