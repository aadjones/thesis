\chapter[Compositional exploration of fluid subspaces]{Compositional explorations of fluid subspaces}

\section{Mode Isolation}
The simplest compositional parameter of interest is the activation or deactivation of individual modes. While a physically accurate subspace re-simulation will in general, at each time step, require a linear combination of each of the $r = 150$ modes, it is also possible to evolve the velocity fields over time according to algorithmic rules rather than physics-based rules. We imagine the $r$ modes as a sort of configuration space, so that a corresponding set of $r$ weights maps to a particular spatial and aural phenomenon. As such, the most elementary experiment is the sequential activation of one mode at a time, creating a corresponding fluid shape of `vibration' which is mapped to its related `frequency' according to the system explained in Chapter 5.

\section{Mode Superposition}
The next experiment to try is the superposition of modes, creating mixtures of the modal vibration shapes in the spatial domain and harmonies in the audio domain. Again, the physics-based time evolution governs a complex coupling of the modal weights that resists simple exploration, so we turn to simple algorithmic rules to better understand the system. We can see the result of mixing a low-, medium-, and high-frequency mode with equal proportions.

\section{Dynamic Control}
